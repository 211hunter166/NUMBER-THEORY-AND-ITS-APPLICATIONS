\documentclass[12pt]{amsart}

\addtolength{\hoffset}{-2.25cm}
\addtolength{\textwidth}{4.5cm}
\addtolength{\voffset}{-2.5cm}
\addtolength{\textheight}{5cm}
\setlength{\parskip}{0pt}
\setlength{\parindent}{15pt}

\usepackage[dvipsnames]{xcolor}

\renewcommand{\baselinestretch}{1.5} 
\definecolor{gray}{HTML}{C7C8C9}

\usepackage{setspace}
\usepackage{amsthm}
\usepackage{amsmath}
\usepackage{amssymb}
\usepackage[colorlinks = true, linkcolor = black, citecolor = black, final]{hyperref}
\usepackage{graphicx}
\usepackage{multicol}
\usepackage{color}
\usepackage{ marvosym }
\usepackage{wasysym}
\newcommand{\ds}{\displaystyle}
\DeclareMathOperator{\sech}{sech}

\setlength{\parindent}{0in}
\pagestyle{empty}
\usepackage [english]{babel}
\usepackage [autostyle, english = american]{csquotes}
\MakeOuterQuote{"}
\begin{document}

\thispagestyle{empty}
\begin{center}
    {\scshape \large  Assignment 1}
\end{center}
{\scshape Numbers Theory and Applications} \hfill 
\hfill {\scshape Vikas Yadav}
\linebreak
{\scshape Project \#4 2022}  \hfill {\scshape Roll: 211166}
\smallskip
\hrule
\bigskip
% First start

\paragraph*{1} Question
\bigskip \\
\textbf{Sol.}
Given $n>0$ \\for n=1, $3^{3n+3} - 26n -27=   676  $,which is divisible by 169. \\
for n=2, $ 3^{3n+3}  - 26n -27=   19604 $,which is divisible by 169.\\
for n$>$2 , $
             3^{3n+3} - 26n -27 = 27^{n + 1}-26n -27\\
             \implies 27^{n + 1}-26n -27= (26 + 1)^{n+ 1}- 26n- 27
            $ 
           \\ Now using binomial expansion, we get\\
            $
            27\times(26 + 1)^{n}- 26n- 27=
            27\times
            ($\dbinom{n}{0}$\times(26)^{n} + $\dbinom{n}{1}$\times(26)^{n-1} + $\dbinom{n}{2}$\times(26)^{n-2}
            $ . + . + . + 
            $
           $\dbinom{n}{n-2}$\times(26)^{2} + 
            $\dbinom{n}{n-1}$\times(26)^{1}  + $\dbinom{n}{n}$\times(26)^{0} 
            $ ) 
            $- 26n - 27$\\
            \implies $
            27\times(26+ 1)^{n}- 26n-27 = 169\times
           \{   27\times4\times   ($\dbinom{n}{0}$\times(26)^{n-2} + $\dbinom{n}{1}$\times(26)^{n-3} + $\dbinom{n}{2}$\times(26)^{n-4}
            $ . + . + . + 
            $
           $\dbinom{n}{n-2}$\times(26)^{0} 
            $ ) + 4n \}
            $
            $
            \\
            \\
            \implies 169|($3^{3n+3} - 26n -27$) \textbf {Answer.}

% First end

% First start

\paragraph*{2} Question\\
\bigskip
\textbf{Sol.} Given $n>0$ ,Using binomial expansion, we get\\
$(n+ 1)^{n}- 1= \dbinom{n}{0}\times(n)^{n} + 
\dbinom{n}{1}\times(n)^{n-1} + . + . + \dbinom{n}{n-2}\times(n)^{2} + \dbinom{n}{n-1}\times(n)^{1} + 
\dbinom{n}{n}\times(n)^{0} - 1
$ \\
\implies $
(n+ 1)^{n}- 1= n^{2}\times\{\dbinom{n}{0}\times(n)^{n-2} + 
\dbinom{n}{1}\times(n)^{n-3} + . + . + \dbinom{n}{n-2}\times(n)^{0}\}
$\\
 \implies $ n^{2}|((n+1)^{n} - 1) $\textbf { Answer.}
% First end
% First start

\paragraph*{3} Question
\bigskip
\textbf{Sol.} By considering all cases, \\
\textbf{CASE 1}: When a is multiple of 7 and b is not a multiple of 7\\
\text{Let a=7n and b=7m + r, where r }\in\text{\{1,2,3\} and n,m } \in \mathbf{Z}\\
$a^{2} + b^{2} = 49\times n^{2} + 49\times m^{2} + 14\times m\times r + r^{2}$\\
\implies $a^{2} + b^{2} = 7\times k + p$\text{ where p }\in\text{\{1,2,3\} and k } \in \mathbf{Z}\\
\text{This  case does not satisfy the given condition.Hence the assumed values of a and b are not possible. }\\
\text{\textbf{CASE 2}: When a is not multiple of 7 and b is a multiple of 7 }\\
\text{Let a=7n+r and b=7m , where r }\in\text{\{1,2,3\} and n,m } \in \mathbf{Z}\\
\textbf{Similar to the previous case.}\\
\text{\textbf{CASE 3}: When a is not multiple of 7 and b is  also not a multiple of 7}\\
\text{Let a=7n+ c and b=7m + r, where r,c }\in\text{\{1,2,3\} and n,m } \in \mathbf{Z}\\
$a^{2} + b^{2} = 49\times n^{2}+14\times n\times c + c^{2} + 49\times m^{2} + 14\times m\times r + r^{2}$\\
\implies $a^{2} + b^{2} = 7\times k + c^{2} + r^{2}$\text{ where c,r }\in\text{\{1,2,3\} and k } \in \mathbf{Z}\\
\text{Considering all possible combinations of r and c, we get}\\
\implies $a^{2} + b^{2} = 7\times k + p$\text{ where p }\in\text{\{2,5,8,10,13,18\} and k } \in \mathbf{Z}\\
\implies $a^{2} + b^{2} = 7\times k + p$\text{ where p }\in\text{\{1,2,3\} and k } \in \mathbf{Z}\\
\text{This  case does not satisfy the given condition.Hence the assumed values of a and b are not possible. }\\
\text{\textbf{CASE 4}: When a is multiple of 7 and b is also  a multiple of 7}\\
\text{Let a=7n and b=7m,}\in\text{ n,m } \in \mathbf{Z}\\
\text{It is the only possible case which satisfies the given condition. }\\
\textbf{Hence 7 divides  both a and b} \text{.Answer}

% First end
% First start

\paragraph*{4} Question
\bigskip
\textbf{Sol.} Rearranging the terms, we get the following equation\\
$2\times(9k + 4)-9\times(2k- 1)= 17$\\ 
(according to linear Diophantine equations, 17 must be a multiple of gcd of 2k-1 and 9k+4)\\
hence, two possible gcd of given numbers are 1 or 17.\\
let s= \{\dfrac{17n+1}{2}  |  n \in (2m+1 | m\in \mathbb{Z})  \} \\
gcd = 17 , \text{ when k }\in{s}\\
gcd = 1 , \text{when k }\not \in{s} \textbf{.Answer}

% First end
% First start

\paragraph*{5} Question
\bigskip
\textbf{Sol.} $
2^{81}=2\times2^{80}=2\times16^{20}=2\times(17-1)^{20}
$
\\Using binomial theorem, We get\\
$
2\times(17-1)^{20}=2\times\{
\dbinom{20}{0}\times17^{20} + \dbinom{20}{1}\times17^{19} (-1)^{1} + ..... +
\dbinom{20}{19}\times17^{1} (-1)^{19} +
\dbinom{20}{20}\times17^{0} (-1)^{20}
\}
$\\
\implies $
2\times(17-1)^{20}=2\times\{
17\times k + 1
\}
$
for some k \in \mathbb{Z}\\
Hence ,
$2^{81} \mod 17= 2$ (Remainder) \textbf{.Answer}
% First end
% First start

\paragraph*{6} Question
\bigskip
\textbf{Sol.} Given $n>0$ \\
$
2^{n} + 6\times9^{n}=2^{n} + 6\times(7+ 2)^{n}=2^{n} + 6\times(7k+ 2^{n})
$  ,for some k \in \mathbb{Z}\\
\implies$ 2^{n} + 6\times9^{n} =7\times(k+ 2^{n})=7\times m 
$ ,for some m \in \mathbb{Z}\\
Hence, $2^{n} + 6\times9^{n}$ is always divisible by 7. \textbf{Answers}\\
% First end
% First start

\paragraph*{7} Question
\bigskip
\textbf{Sol.}  We can determine if our number is divisible by $3$ or $9$ by summing the digits. Looking at the one's place, we can start out with $0, 1, 2, 3, 4, 5, 6, 7, 8, 9$ and continue cycling though the numbers from $0$ through $9$. For each one of these cycles, we add $0 + 1 + ... + 9 = 45$. This is divisible by $9$, thus we can ignore the sum. However, this excludes $19$, $90$, $91$ and $92$. These remaining units digits sum up to $9 + 1 + 2 = 12$, which means our units sum is $3 \pmod 9$. As for the tens digits, for $2, 3, 4, \cdots , 8$ we have $10$ sets of those:\[\frac{8 \cdot 9}{2} - 1 = 35,\]which is congruent to $8 \pmod 9$. We again have $19, 90, 91$ and $92$, so we must add\[1 + 9 \cdot 3 = 28\]to our total. $28$ is congruent to $1 \pmod 9$. Thus our sum is congruent to $3 \pmod 9$, and hence $K=1$ \textbf{.Answer}

% First end
% First start

\paragraph*{8} Question
\bigskip
\textbf{Sol.} 
Given z$>$1
\\We have\\
$x^{2} + y^{2}=10^{z} - 1.$
where x,y \in \mathbb{I}  \\
\text{RHS is of the form (99999.....z times) for any z$>$1}\\
\text{So LHS also needs to be odd , without loss of generality }\\ \text{we assume }
$x=2m and y=2n+1$ \text{where n,m} \in \mathbb{Z}\\
\implies $4\times m^{2} + 4\times n^{2} + 4n + 1= 2^{z}\times5^{z} - 1$ \\
\implies $4\times (m^{2} + n^{2} + n )= 2^{z}\times5^{z} - 2$ \\
\implies $4\times (m^{2} + n^{2} + n )= 4\times k - 2$ \text{as z}\geq{2}\text{ and k }\in \mathbb{Z}\\
\implies $4\times (m^{2} + n^{2} + n ) \not = 4\times k - 2$ \text{ as m,n and k are integers}\\
\text {This is a contradiction to our assumption.}\\ \text{Hence there does not exist any n and m for which given equation has a solution} \textbf{.Answer}


% First end
% First start

\paragraph*{9*} Question
\bigskip
\textbf{Sol.} Using C\\
\#include$<$stdio.h$>$ \\
int main () \{ \\
int a,b,r$=$-1;\\
scanf(" \%d " , \&a);\\
scanf" \%d " , \&b);\\
if (a$>$b) \{ \\
int t$=$0;
t$=$a;
a$=$b;
b$=$t;
\} \\
while(1) \{ \\
r= b\%a ;\\
if (r$=$$=$0) break;\\
b$=$a;\\
a$=$r;\\
\} \\
printf("\%d", \&a);\\
 return 0;\\
\}
% First end
% First start

\paragraph*{10*} Question
\bigskip
\textbf{Sol.} \textbf{WORK IN PROGRESS}\\
\textbf{TOO LENGTHY TO WRITE IN LATEX WITH PROPER SYMBOLS }\\
\\Using C\\
\#include$<$stdio.h$>$ \\
int main () \{ \\
int a,b,r$=$-1,gcd,A,B;\\
scanf(" \%d " , \&a);\\
scanf" \%d " , \&b);\\
if (a$>$b) \{ \\
int t$=$0;\\
t$=$a;\\
a$=$b;\\
b$=$t;\\
\} \\
A$=$a;\\
B$=$b;\\
while(1) \{ \\
r= b\%a ;\\
if (r$=$$=$0) break;\\
b$=$a;\\
a$=$r;\\
\} \\
gcd$=$a;\\
 return 0;\\
\}

% First end

\end{document}