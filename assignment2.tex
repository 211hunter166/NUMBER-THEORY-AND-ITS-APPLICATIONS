\documentclass[12pt]{amsart}

\addtolength{\hoffset}{-2.25cm}
\addtolength{\textwidth}{4.5cm}
\addtolength{\voffset}{-2.5cm}
\addtolength{\textheight}{5cm}
\setlength{\parskip}{0pt}
\setlength{\parindent}{15pt}

\usepackage[dvipsnames]{xcolor}

\renewcommand{\baselinestretch}{1.5} 
\definecolor{gray}{HTML}{C7C8C9}

\usepackage{setspace}
\usepackage{amsthm}
\usepackage{amsmath}
\usepackage{amssymb}
\usepackage[colorlinks = true, linkcolor = black, citecolor = black, final]{hyperref}
\usepackage{graphicx}
\usepackage{multicol}
\usepackage{color}
\usepackage{ marvosym }
\usepackage{wasysym}
\newcommand{\ds}{\displaystyle}
\DeclareMathOperator{\sech}{sech}

\setlength{\parindent}{0in}
\pagestyle{empty}
\usepackage [english]{babel}
\usepackage [autostyle, english = american]{csquotes}
\MakeOuterQuote{"}
\begin{document}

\thispagestyle{empty}
\begin{center}
    {\scshape \large  Assignment 2}
\end{center}
{\scshape Numbers Theory and Applications} \hfill 
\hfill {\scshape Vikas Yadav}
\linebreak
{\scshape Project \#4 2022}  \hfill {\scshape Roll: 211166}
\smallskip
\hrule
\bigskip
% First start

\paragraph*{1} Question
\bigskip \\
\textbf{Sol.}
\text{from the equations given in question}\\
x$=3k + 2$ \text{for some k} \in  \mathbf{Z}   \text{      ...(i)}\\
\text{putting it in 2nd equation, we get}\\
$3k + 2= 3(mod 5)$   \\
 $3k = 1(mod 5)$\\ 
$2.3k= 2.1(mod 5)$ \iff $k= 2(mod 5)$\\
\implies $k= 5j +2$  \text{ for some j} \in  \mathbf{Z}   \text{      ...(ii)}\\
\text{from equation (i) and (ii) , we get}\\
x $= 8 + 15j$  \text{ for some j} \in  \mathbf{Z}   \text{      ...(iii)}\\
\text{now using  eqn (iii)}\\
$8 + 15j = 2(mod 7)$ \iff  $1 + j = 2(mod 7)$\\
\implies $ j = 1(mod 7)$\\
\implies $ j = 7i + 1$ \text{ for some i} \in  \mathbf{Z} \text{      ...(iv)} \\
\text{from (iii) and (iv), we get}\\
x$ = 23 + 105i $ \text{for some i} \in  \mathbf{Z}\\
\implies x$= 23(mod 105)$
\textbf{ Answer.}
\bigskip
% First end

% First start

\paragraph*{2} Question\\
\bigskip
\textbf{Sol.}
\text{from the equations given in question}\\
x$=36k + 11$ \text{for some k} \in  \mathbf{Z}   \text{      ...(i)}\\
\text{putting it in 2nd equation, we get}\\
$36k + 11= 7(mod 40)$   \\
 $36k = -4(mod 40)$ \iff $36k = -4 + 40a $ \iff $9k = -1 + 10a $ \iff $9k = -1(mod 10) $  \text{ for some a} \in  \mathbf{Z}\\ 
 \implies $9k = 9(mod 10) $ \iff $9.9k = 9.9(mod 10) $ \iff $k = 1(mod 10) $ \\ 
 \implies $k = 1 +10j $\text{ for some j} \in  \mathbf{Z} \text{      ...(ii)}\\
 \text{from (i) and (ii), we get,}\\
 $x= 47 + 360j$ \text{ for some j} \in  \mathbf{Z} \text{      ...(iii)}\\
\text{now using  eqn (iii) and the unused equation in question }\\
$47 + 360j = 32(mod 75)$ \iff  $360j = -15(mod 75)$ \iff  $24j = -1(mod 5)$\\
\implies $ 4j = 4(mod 5)$ \iff $ 9.4j = 9.4(mod 5)$  \iff $ j = 1(mod 5)$\\
\implies $ j = 5i + 1$ \text{ for some i} \in  \mathbf{Z} \text{      ...(iv)} \\
\text{from (iii) and (iv), we get}\\
x$ = 407+ 1800i $ \text{for some i} \in  \mathbf{Z}\\
\implies x$= 407(mod 1800)$
\textbf{ Answer.}
% First end
% First start
\bigskip
\paragraph*{3} Question
\bigskip
\textbf{Sol.} $x^{2} = $1(mod3)  \iff  $x^{2} - 1= $0(mod3) \\
\implies \text{3 divides either x-1 or x+1 (cannot divide both as their difference is 2)}\\
\textbf{CASE 1} \text{ When 3 divides x-1}\\
\implies $x=1 (mod 3) $\text{ also x$ = 2(mod 4)$ as given in question. }\\ \text{now, solving the above equations,we get}\\
x$=10$(mod12) \iff x$=-2$(mod12) \text{      ...(i)}\\

\textbf{CASE 2} \text{ When 3 divides x+1}\\
\implies $x= -1 (mod 3)$ \equiv$ 2(mod 3)$\text{ also x$ = 2(mod 4)$ as given in question. }\\ \text{now, solving the above equations,we get}\\
x$=2$(mod12) \text{      ...(ii)}\\
\text{Hence x$=-2$(mod12) or  x$=2$(mod12)}\\
\implies $x^{2}= 4(mod 12)$
\textbf{.Answer}

% First end
% First start
\bigskip
\paragraph*{4} Question
\bigskip
\textbf{Sol.} \textbf{.Answer}

% First end
% First start

\paragraph*{5} Question
\bigskip
\textbf{Sol.}
\text{any no x} \in \text{p-safe if and only if x}\ge\text{3(mod p) and x}\le\text{(p-3)(mod p)}\\
\text{where p}\in {(7,11,13)} \\
$Therefore, a number $x$ satisfying the conditions can have $2$ different residues ($\mod 7$), $6$ different residues ($\mod 11$), and $8$ different residues ($\mod 13$) $\\
$According to chinese remainder theorem a number $x$ that is $a$ (mod b) $c$ (mod d) $e$ (mod f) has one solution if $gcd(b,d,f)=1$.In our case, the number $x$ can be: 3 (mod 7) 3 (mod 11) 7 (mod 13) so since $gcd(7,11,13)$=1, there is 1 solution for x for this case of residues of $x$. $\\
\implies$ $x$ can have $2\cdot 6 \cdot 8 = 96$ different residues mod ($7 \cdot 11 \cdot 13 = 1001$ )$\\
\implies $x can have 960 values satisfying the conditions in the range $0 < n \le 10010$ $\\
 $However, we must now remove any values greater than $10000$ that satisfy the above conditions. By checking , we see that the such values are $10006$ and $10007$, so required no is $\fbox{958}$ $
\textbf{.Answer}
% First end
% First start
\bigskip
\paragraph*{6} Question
\bigskip
\textbf{Sol.} \textbf{Answers}\\
% First end
% First start
\end{document}