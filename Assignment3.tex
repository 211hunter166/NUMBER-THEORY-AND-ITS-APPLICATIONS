\documentclass[12pt]{amsart}

\addtolength{\hoffset}{-2.25cm}
\addtolength{\textwidth}{4.5cm}
\addtolength{\voffset}{-2.5cm}
\addtolength{\textheight}{5cm}
\setlength{\parskip}{0pt}
\setlength{\parindent}{15pt}

\usepackage[dvipsnames]{xcolor}

\renewcommand{\baselinestretch}{1.5} 
\definecolor{gray}{HTML}{C7C8C9}

\usepackage{setspace}
\usepackage{amsthm}
\usepackage{amsmath}
\usepackage{amssymb}
\usepackage[colorlinks = true, linkcolor = black, citecolor = black, final]{hyperref}
\usepackage{graphicx}
\usepackage{multicol}
\usepackage{color}
\usepackage{ marvosym }
\usepackage{wasysym}
\newcommand{\ds}{\displaystyle}
\DeclareMathOperator{\sech}{sech}

\setlength{\parindent}{0in}
\pagestyle{empty}
\usepackage [english]{babel}
\usepackage [autostyle, english = american]{csquotes}
\MakeOuterQuote{"}
\begin{document}

\thispagestyle{empty}
\begin{center}
    {\scshape \large  Assignment 3}
\end{center}
{\scshape Numbers Theory and Applications} \hfill 
\hfill {\scshape Vikas Yadav}
\linebreak
{\scshape Project \#4 2022}  \hfill {\scshape Roll: 211166}
\smallskip
\hrule
\bigskip
% First start

\paragraph*{1} Question
\bigskip \\
\textbf{Sol.}
Given statement is equivalent to $2^{2^{6k + 2}} \equiv 16 \pmod{19}$ \\
We will go via induction method \\
Base case for: $k=0$ we get $2^{4}\equiv 16 \pmod{19}$\\
let the $2^{2^{6m + 2}} \equiv 16 \pmod{19}$ for some m $>$ 0\\
now we will prove it for m$+$1\\
$2^{2^{6(m+ 1) + 2}} $\equiv $2^{2^{6m + 8}}$ \equiv $2^{2^{6m+ 2}}.2^{6}$ \equiv $(2^{2^{6m + 2}})^{64}$ \equiv $16^{64}$\\
Using fermit little theorem ,\\
$16^{64}$\equiv $(-3)^{64}$ \equiv $(-3)^{18\times 3+10}$ \pmod{19} \equiv $(-3)^{10}$\equiv $(-3)^{10}$ \pmod{19}\equiv $(-3).(-27)^{3}$ \pmod{19}\equiv $(-3).(-8)^{3}$ \pmod{19}\equiv $(24).(64)$ \pmod{19}\equiv $35$ \pmod{19}\equiv $16$ \pmod{19} \boxed{PROVED}


\textbf{ Answer.}
\bigskip
% First end

% First start

\paragraph*{2} Question\\
\bigskip
\textbf{Sol.}
\textbf{ Answer.}
% First end
% First start
\bigskip
\paragraph*{3} Question
\bigskip
\textbf{Sol.}
Given $S(n)=1^n+2^n+3^n+⋯+($n − 1$)n ,  n>1 .$
\\
concept used :If  n  is odd and  a,b  are positive integers, then  $(a+b)∣(a^n+b^n$) .\\Now rearranging the given equation
$  $ $1^{n}+($n-1$)^n ,  2^n+($n-2$)^n ,  3^n+($n-3$)^n , … ,  (($(n+1)/2$)^n +($(n-1)/2$)^n $. Hence  n  also divides their sum, which is  S(n) .

Let  n  be even, and let $ 2^e $ be the highest power of  2  dividing  n . We claim that  $S(n)\equiv$(1/2)*n\pmod{$2^e$}…(1) $ 

Each of the terms \mathrm{2^n$ ,  $4^n$ ,  $6^n$ ,  … ,  $($n-1$)^n} \text{ is a multiple of } $2^n$, and so also of  $2^e$  since $$ e<(1+1)e\leq n $\text {On the other hand, for each k}\in { 
\text{1,3,5,…,  } n- 1}
,   $k\phi(2^e)\equiv1\pmod{2^e} \text{by Euler’s theorem. Since }  \phi (2^e) =$ \mathrm{ 2^{n−1} } \text{ divides
n}  ,\text{ it follows that } k\phi ($n-1$)\equiv 1 \pmod{2^{e} } \text{ for each such  k . There are  } $(1/2).n $ \text {integers in  {1,3,5,…,n−1} , resulting in our claim given as  (1) .}

Now if  n divides S(n) , then  2^e divides S(n$) . But then  $2^e∣(1/2)n$ , and$  2^{e+1}   $, which contradicts the definition of  e  given above.

\text{We conclude that  n divides S(n )}\iff \text{ n  is odd},  n\geq1
\textbf{.Answer}

% First end
% First start
\bigskip
\paragraph*{4} Question
\bigskip
\textbf{Sol.} \textbf{.Answer}

% First end
% First start

\paragraph*{5} Question
\bigskip
\textbf{Sol.}
Question statement translates to finding a,b,c given by following equations\\
$n=2.a^{2}, n=3.a^{3},n=5.a^{5}$ \ldots (i)\\
\implies 2|n , 3|n , 5|n\\
\text{a is a multiple of 15}\\
\text{b is a multiple of 10}\\
\text{c is a multiple of 6}\\
\text{now the question translates to finding c1 ,c2 and c3 that satisfy the given equations}\\

$n=2.3^{2}5^{2}.(c1)^{2}, n=3.2^{3}.5^{3}.(c2)^{3}, n=5.2^{5}.3^{5}.(c3)^{5}$ \ldots (ii) \\
\text{equating the index of 2 3 and 5 to get one possible answer}\\
$c1=2^{7}.3^{4}.5^{2},\\c2=2^{6}.3^{3}.5^{1},\\c3=2^{2}.3^{1}.5^{1},\\N=2^{15}.3^{10}.5^{6}, $\\
\text{general soln}\\
$c1=2^{7}.3^{4}.5^{2}.k^{15},\\c2=2^{6}.3^{3}.5^{1}.k^{10},\\c3=2^{2}.3^{1}.5^{1}.k^{6},\\N=2^{15}.3^{10}.5^{6}.k^{30}, $ \text{where k is any integer}
\textbf{.Answer}
% First end
% First start
\bigskip
\paragraph*{6} Question
\bigskip
\textbf{Sol.} \\
Taking  modulo $2,3,$ and $5,$ respectively, we have\begin{align*} n^5&\equiv0\pmod{2}, \\ n^5&\equiv0\pmod{3}, \\ n^5&\equiv4\pmod{5}. \end{align*}By either Fermat's Little Theorem  we get\begin{align*} n&\equiv0\pmod{2}, \\ n&\equiv0\pmod{3}, \\ n&\equiv4\pmod{5}. \end{align*}By either the Chinese Remainder Theorem (CRT)  we get $n\equiv24\pmod{30}.$

It is clear that $n>133,$ so the possible values for $n$ are $144,174,204,\ldots.$ Note that\begin{align*} n^5&=133^5+110^5+84^5+27^5 \\ &<133^5+110^5+(84+27)^5 \\ &=133^5+110^5+111^5 \\ &<3\cdot133^5, \end{align*}from which $\left(\frac{n}{133}\right)^5<3.$

If $n\geq174,$ then\begin{align*} \left(\frac{n}{133}\right)^5&>1.3^5 \\ &=1.3^2\cdot1.3^2\cdot1.3 \\ &>1.6\cdot1.6\cdot1.3 \\ &=2.56\cdot1.3 \\ &>2.5\cdot1.2 \\ &=3, \end{align*}which arrives at a contradiction. Therefore, we conclude that $n={144}.$
\textbf{Answers}\\
% First end
% First start
\bigskip
\paragraph*{7} Question
\bigskip
\textbf{Sol.}
\\
Let $
L= \sum_{i=1}^{9}{\frac{1}{i}}$.  terms in $S_1$, we see that $S_1 = L + 1$ since each digit $n$ appears once and 1 appears an extra time. Now consider writing out $S_2$. Each term of $L$ will appear 10 times in the units place and 10 times in the tens place (plus one extra 1 will appear), so $S_2 = 20L + 1$.

In general, we will have that

$S_n = (n10^{n-1})L + 1$
because each digit will appear $10^{n - 1}$ times in each place in the numbers $1, 2, \ldots, 10^{n} - 1$, and there are $n$ total places.

The denominator of $L$ is $D = 2^3\cdot 3^2\cdot 5\cdot 7$. For $S_n$ to be an integer, $n10^{n-1}$ must be divisible by $D$. Since $10^{n-1}$ only contains the factors $2$ and $5$ (but will contain enough of them when $n \geq 3$), we must choose $n$ to be divisible by $3^2\cdot 7$. Since we're looking for the smallest such $n$, the answer is ${063}$ .\textbf{Answers}\\
% First end
% First start
\bigskip
\paragraph*{8} Question
\bigskip
\textbf{Sol.} \\
\begin{verbatim}
#include<stdio.h>
#include<stdlib.h>
#include<math.h>

int gcd(int a,int b){
    if (a == 0)
       return b;
    if (b == 0)
       return a;
 
    // base case
    if (a == b)
        return a;
 
    // a is greater
    if (a > b)
        return gcd(a-b, b);
    return gcd(a, b-a);
}

int coprime(int n,int i){
    if (gcd(n,i)==1) return 1;
    else return 0;
}

int main(){
    long long int n=0;
    scanf("%lld",&n);

    int ct=0;
    for (int i=0; i<n ;i++){
        if (coprime(n,i)==1) ct++;
    }
    printf("%d",ct);
    
    return 0;
}
\end{verbatim}
\\\textbf{Answers}\\
% First end
% First start
\bigskip
\paragraph*{9} Question
\bigskip
\textbf{Sol.} \\
\begin{verbatim}
#include<stdio.h>
#include<stdlib.h>
#include<math.h>

int gcd(int a,int b){
    if (a == 0)
       return b;
    if (b == 0)
       return a;
 
    // base case
    if (a == b)
        return a;
 
    // a is greater
    if (a > b)
        return gcd(a-b, b);
    return gcd(a, b-a);
}

int coprime(int n,int i){
    if (gcd(n,i)==1) return 1;
    else return 0;
}

int main(){
    long long int n=0;
    scanf("%lld",&n);
    for (int m=1;m<=n;m++){
    int ct=0;
    for (int i=0; i<m ;i++){
        if (coprime(m,i)==1) ct++;
    }
    printf("%d\n",ct);
    }
    
    return 0;
}
 
\end{verbatim}
\\
\textbf{Answers}\\
% First end
% First start
\bigskip
\paragraph*{10} Question
\bigskip
\textbf{Sol.} \textbf{Answers}\\
% First end
% First start
\bigskip
\paragraph*{11} Question
\bigskip
\textbf{Sol.} \\
\begin{verbatim}
#include<stdio.h>
#include<stdlib.h>
#include<math.h>
int main(){
    long long int n=0;
    scanf("%lld",&n);
    int* arr=(int *)malloc(n*(sizeof(long long int)));
    for (int i=0;i<n;i++){
        arr[i]=1;
    }
    for (int i=2;i<=sqrt(n)+1;i++){
        for (int j=i*i;j<=n;j++){
            if (j%i == 0) arr[j]=0;
        }
    }
    for (int i=2;i<=n;i++){
        if (arr[i])
        printf("%d \n",i);
    }
    
    return 0;
}
\end{verbatim}

\textbf{Answers}\\
% First end
% First start
\end{document}